\documentclass[letterpaper,headings=standardclasses]{scrartcl}

\usepackage[margin=1in,includefoot]{geometry}
\usepackage{algorithmicx}
\usepackage{algpseudocode}
\usepackage{amssymb}
\usepackage{amsmath}

\title{Homework 8}
\subtitle{CS 411 - Artificial Intelligence I - Fall 2019}
\author{Matteo Corain 650088272}

\begin{document}

\maketitle

\section{Question 1}

\subsection{Knowledge base}

A \emph{knowledge base} is a set of sentences (in conjunction with each other), each of which is expressed in a proper knowledge representation language, that represent some assertions that describe the current state of the world. Each sentence in a knowledge base represent a piece of information that is known to be true in the particular world we are considering.

\subsection{Inference}

\emph{Inference} is the process of deriving a new sentence $\alpha$ from the ones that we already know to hold true (components of the knowledge base), through a certain procedure $i$. The inference operation is formally denoted as:

$$ KB \vdash_i \alpha $$

It is desirable that procedure $i$ satisfies two requirements:

\begin{itemize}
    \item \emph{Soundness}: an inference procedure is \emph{sound} if,	whenever $ KB \vdash_i \alpha $, it is also true that $ KB \models \alpha $ (it can entail any sentence that is derived);
    \item \emph{Completeness}: an inference procedure is \emph{complete} if, whenever $ KB \models \alpha $, it is also true that $ KB \vdash_i \alpha $ (it can derive any sentence that is entailed).
\end{itemize}

Examples of procedures that allow to perform inference include \emph{inference by enumeration} (model checking), \emph{modus ponens}, and \emph{resolution}.

\subsection{Model}

A \emph{model} is the formalization of the concept of a ``possible world'', represented by a mathematical abstraction that describes it by fixing the truth value of relevant primitive sentences.

The knowledge base $KB$ of a knowledge-based agent may satisfy not just a single model, but instead a \emph{set of models}, which is denoted as $M(KB)$. When the knowledge base is made up of few sentences, the cardinality of the set of models may be huge; the more knowledge the agent acquires, the more the set of models shrinks.

\subsection{Entailment}

The concept of \emph{entailment} formalizes the idea of a sentence $\alpha$ following logically from a knowledge base $KB$ and it is denoted with the syntax:

$$ KB \models \alpha $$

Formally, $ KB \models \alpha $ if and only if, in each model in which $KB$ is true, then $\alpha$ is also true. In terms of sets of models, this means that:

$$ KB \models \alpha \Leftrightarrow M(KB) \subseteq M(\alpha) $$

In fact, since $KB$ represents a stronger assumption than $\alpha$, this means that the models that $KB$ satisfies are in general less than the models that $\alpha$ satisfies. The existence of an entailment relationship may be simply verified using the procedure known as \emph{model checking}: this requires to enumerate all possible models for the sentences in the knowledge base and for sentence $\alpha$ and to verify whether the inclusion relation $ M(KB) \subseteq M(\alpha) $ holds ($\alpha$ is true in all worlds in which $KB$ is true).

\subsection{Valid sentence}

A sentence is valid (or, equivalently, \emph{tautological}) if it holds true in all possible models; in other words, every valid sentence is inconditionally logically equivalent to true.

Valid sentences are useful for deriving entailment relationships, according to the \emph{deduction theorem}: for any sentences $\alpha$ and $\beta$, $\alpha \models \beta$ if and only if the sentence $\alpha \Rightarrow \beta$ is valid.

\section{Question 2}

\subsection{Knowledge base representation}

Let us consider the following propositional symbols for the atomic sentences that appear in the knowledge base:

\begin{itemize}
    \item $X$: ``Sam plays baseball'';
    \item $Y$: ``Paul plays baseball'';
    \item $Z$: ``Ryan plays baseball''.
\end{itemize}

According to propositional logic, the knowledge base in question may be represented by the conjunction of two sentences which hold true in the considered world:

\begin{itemize}
    \item ``Sam plays baseball or Paul plays baseball'': this can be represented as a disjunction between sentences $X$ and $Y$:
    $$ X \vee Y $$
    \item ``Sam plays baseball or Ryan doesn't play baseball'': this can be represented as a disjunction between sentences $X$ and $\neg Z$:
    $$ X \vee \neg Z $$
\end{itemize}

The overall knowledge base is therefore represented by the expression:

$$ KB = (X \vee Y) \wedge (X \vee \neg Z) = X \wedge (Y \vee \neg Z) $$

The truth table for such a knowledge base is shown in table \ref{tt_kb}.

\begin{table}[h]
    \centering
    \begin{tabular}{|c|c|c|c|c|c|}
    \hline
    $X$ & $Y$ & $Z$ & $X \vee Y$ & $X \vee \neg Z$ & $KB$ \\ \hline
    false & false & false & false & true & false \\ \hline
    false & false & true & false & false & false \\ \hline
    false & true & false & true & true & true \\ \hline
    false & true & true & true & false & false \\ \hline
    true & false & false & true & true & true \\ \hline
    true & false & true & true & true & true \\ \hline
    true & true & false & true & true & true \\ \hline
    true & true & true & true & true & true \\ \hline
    \end{tabular}
    \caption{Truth table for the knowledge base}
    \label{tt_kb}
\end{table}

\subsection{Entailment A}

The sentence ``Sam and Ryan both play baseball'' may be represented in terms of the atomic propositions $X$ and $Z$ as:

$$ A = X \wedge Z $$

To check whether this sentence is entailed by the knowledge base, we can use a model checking procedure, as shown in table \ref{tt_a}.

\begin{table}[h]
    \centering
    \begin{tabular}{|c|c|c|c|c|}
    \hline
    $X$ & $Y$ & $Z$ & $KB$ & $A = X \wedge Z$ \\ \hline
    false & false & false & false & false \\ \hline
    false & false & true & false & false \\ \hline
    false & true & false & true & false \\ \hline
    false & true & true & false & false \\ \hline
    true & false & false & true & false \\ \hline
    true & false & true & true & true \\ \hline
    true & true & false & true & false \\ \hline
    true & true & true & true & true \\ \hline
    \end{tabular}
    \caption{Model checking for entailment A}
    \label{tt_a}
\end{table}

Since there are worlds in which the knowledge base holds true but the presented sentence does not, we cannot state that $A$ can be entailed from the knowledge base. For example, in a model in which Paul plays baseball but the other two do not, the knowledge base holds true but the given sentence does not (neither Sam nor Ryan play baseball).

\subsection{Entailment B}

The sentence ``At least one among Sam, Paul and Ryan play baseball'' may be represented in terms of the atomic propositions $X$, $Y$ and $Z$ as:

$$ B = X \vee Y \vee Z $$

To check whether this sentence is entailed by the knowledge base, we can use a model checking procedure, as shown in table \ref{tt_b}.

\begin{table}[h]
    \centering
    \begin{tabular}{|c|c|c|c|c|}
    \hline
    $X$ & $Y$ & $Z$ & $KB$ & $B = X \vee Y \vee Z$ \\ \hline
    false & false & false & false & false \\ \hline
    false & false & true & false & true \\ \hline
    false & true & false & true & true \\ \hline
    false & true & true & false & true \\ \hline
    true & false & false & true & true \\ \hline
    true & false & true & true & true \\ \hline
    true & true & false & true & true \\ \hline
    true & true & true & true & true \\ \hline
    \end{tabular}
    \caption{Model checking for entailment B}
    \label{tt_b}
\end{table}

Since in every world in which the knowledge base is true so it is also the given sentence, we can state that $B$ can be entailed from the knowledge base.

\end{document}