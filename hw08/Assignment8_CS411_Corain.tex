\documentclass[letterpaper,headings=standardclasses]{scrartcl}

\usepackage[margin=1in,includefoot]{geometry}
\usepackage{algorithmicx}
\usepackage{algpseudocode}
\usepackage{amssymb}
\usepackage{amsmath}

\title{Homework 8}
\subtitle{CS 411 - Artificial Intelligence I - Fall 2019}
\author{Matteo Corain 650088272}

\begin{document}

\maketitle

\section{Question 1}

\subsection{Knowledge base}

\subsection{Inference}

\subsection{Model}

\subsection{Entailment}

\subsection{Valid sentence}

\section{Question 2}

\subsection{Knowledge base representation}

Let us consider the following propositional symbols for the atomic sentences that appear in the knowledge base:

\begin{itemize}
    \item $X$: "Sam plays baseball";
    \item $Y$: "Paul plays baseball";
    \item $Z$: "Ryan plays baseball".
\end{itemize}

According to propositional logic, the knowledge base in question may be represented by the conjunction of two sentences which have to hold true:

\begin{itemize}
    \item "Sam plays baseball or Paul plays baseball": this can be represented as a disjunction between sentences $X$ and $Y$:
    $$ X \vee Y $$
    \item "Sam plays baseball or Ryan doesn't play baseball": this can be represented as a disjunction between sentences $X$ and $\neg Z$:
    $$ X \vee \neg Z $$
\end{itemize}

The overall knowledge base is therefore represented by the expression:

$$ (X \vee Y) \wedge (X \vee \neg Z) $$

Applying logical equivalence rules, this may also be rewritten as:

$$ X \wedge (Y \vee \neg Z) $$

The truth table for such a knowledge base is shown in table \ref{tt_kb}.

\begin{table}[h]
    \centering
    \begin{tabular}{|c|c|c|c|c|c|}
    \hline
    $X$ & $Y$ & $Z$ & $X \vee Y$ & $X \vee \neg Z$ & $KB$ \\ \hline
    false & false & false & false & true & false \\ \hline
    false & false & true & false & false & false \\ \hline
    false & true & false & true & true & true \\ \hline
    false & true & true & true & false & false \\ \hline
    true & false & false & true & true & true \\ \hline
    true & false & true & true & true & true \\ \hline
    true & true & false & true & true & true \\ \hline
    true & true & true & true & true & true \\ \hline
    \end{tabular}
    \caption{Truth table for the knowledge base}
    \label{tt_kb}
\end{table}

\subsection{Entailment A}

The sentence "Sam and Ryan both play baseball" may be represented in terms of the atomic propositions $X$ and $Z$ as:

$$ A = X \wedge Z $$

To check whether this sentence is entailed by the knowledge base, we can use a model checking procedure, as shown in table \ref{tt_a}.

\begin{table}[h]
    \centering
    \begin{tabular}{|c|c|c|c|c|}
    \hline
    $X$ & $Y$ & $Z$ & $KB$ & $A = X \wedge Z$ \\ \hline
    false & false & false & false & false \\ \hline
    false & false & true & false & false \\ \hline
    false & true & false & true & false \\ \hline
    false & true & true & false & false \\ \hline
    true & false & false & true & false \\ \hline
    true & false & true & true & true \\ \hline
    true & true & false & true & false \\ \hline
    true & true & true & true & true \\ \hline
    \end{tabular}
    \caption{Model checking for entailment A}
    \label{tt_a}
\end{table}

Since there are worlds in which the knowledge base holds true but the presented sentence does not, we cannot state that $A$ can be entailed from the knowledge base. For example, in a model in which Paul plays baseball but the other two do not, the knowledge base holds true but the given sentence does not (neither Sam nor Ryan play baseball).

\subsection{Entailment B}

The sentence "At least one among Sam, Paul and Ryan play baseball" may be represented in terms of the atomic propositions $X$, $Y$ and $Z$ as:

$$ B = X \vee Y \vee Z $$

To check whether this sentence is entailed by the knowledge base, we can use a model checking procedure, as shown in table \ref{tt_b}.

\begin{table}[h]
    \centering
    \begin{tabular}{|c|c|c|c|c|}
    \hline
    $X$ & $Y$ & $Z$ & $KB$ & $B = X \vee Y \vee Z$ \\ \hline
    false & false & false & false & false \\ \hline
    false & false & true & false & true \\ \hline
    false & true & false & true & true \\ \hline
    false & true & true & false & true \\ \hline
    true & false & false & true & true \\ \hline
    true & false & true & true & true \\ \hline
    true & true & false & true & true \\ \hline
    true & true & true & true & true \\ \hline
    \end{tabular}
    \caption{Model checking for entailment B}
    \label{tt_b}
\end{table}

Since in every world in which the knowledge base is true so it is also the given sentence, we can state that $B$ can be entailed from the knowledge base.

\end{document}