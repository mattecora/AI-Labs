\documentclass[letterpaper,headings=standardclasses]{scrartcl}

\usepackage[margin=1in,includefoot]{geometry}
\usepackage{algorithmicx}
\usepackage{algpseudocode}
\usepackage{amssymb}
\usepackage{amsmath}

\title{Homework 9}
\subtitle{CS 411 - Artificial Intelligence I - Fall 2019}
\author{Matteo Corain 650088272}

\begin{document}

\maketitle

\section{Question 1}

\subsection{Syntax}

\subsection{Semantics}

\subsection{Modus ponens}

\subsection{Monotonicity}

\subsection{Proof by contradiction}

\section{Question 2}

Let us introduce the following atomic propositions:

\begin{itemize}
    \item $A$: ``Ana eats'';
    \item $B$: ``Bret eats'';
    \item $C$: ``Charles eats'';
    \item $D$: ``Derek eats'';
    \item $E$: ``Earl eats'';
    \item $F$: ``Fred eats''.
\end{itemize}

Using those symbols, it is possible to translate the given sentences in formal propositional sentences in the following way:

\begin{itemize}
    \item ``If Ana eats, Bret eats'': $A \Rightarrow B$;
    \item ``Charles eats and Derek doesn't eat'': $C \wedge \neg D$;
    \item ``Bret doesn't eat'': $\neg B$;
    \item ``If Derek doesn't eat, at least one among Ana, Earl and Fred eats'': $\neg D \Rightarrow (A \vee E \vee F)$;
    \item ``If at least one of Charles and Gary eats, Earl doesn't' eat': $(C \vee G) \Rightarrow \neg E$.
\end{itemize}

\section{Question 3}

In order to prove that $F$ is true, the following steps may be taken:

\begin{itemize}
    \item From sentence 3, we know that $\neg B$ is true, which means that $B$ is necessarily false ($\neg True = False$);
    \item From sentence 2, we know that $C \wedge \neg D$ is true, which means that $C$ is necessarily true and $D$ is necessarily false ($True \wedge \neg False = True$);
    \item From sentence 1, we know that $A \Rightarrow B$ is true; since $B$ is false, then $A$ is necessarily false ($False \Rightarrow False = True$);
    \item From sentence 5, we know that $(C \vee G) \Rightarrow \neg E$ is true; since $C$ is true, then $C \vee G$ is true independently of the value of $G$; therefore, $\neg E$ has to be true ($True \Rightarrow True = True$) and, necessarily $E$ is false;
    \item Finally, from sentence 4, we know that $\neg D \Rightarrow (A \vee E \vee F)$ is true; since $D$ is false, then $\neg D$ is true, which means that $A \vee E \vee F$ needs to be true as well ($True \Rightarrow True = True$); however, since $A$ and $E$ are both false, necessarily $F$ is true.
\end{itemize}

The truth value of the atomic propositions is summarized in table \ref{tt}.

\begin{table}[h]
    \centering
    \begin{tabular}{|c|c|}
    \hline
    Proposition & Value \\ \hline
    $A$ & False \\ \hline
    $B$ & False \\ \hline
    $C$ & True \\ \hline
    $D$ & False \\ \hline
    $E$ & False \\ \hline
    $F$ & True \\ \hline
    $G$ & Undetermined \\ \hline
    \end{tabular}
    \caption{Truth status of the atomic propositions}
    \label{tt}
\end{table}

\end{document}